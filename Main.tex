\documentclass[9pt]{scrreprt}
\usepackage{graphicx} % Required for inserting images
\usepackage[top=2.5cm, bottom=2.5cm, left = 2cm, right = 2cm]{geometry}
\geometry{a4paper}
\usepackage[utf8]{inputenc}
\usepackage{amsmath,amssymb}
\usepackage[pdftex,bookmarks,colorlinks,breaklinks]{hyperref}
\hypersetup{linkcolor=blue,citecolor=red,filecolor=black,urlcolor=black}
%\setlength{\headheight}{12pt}
\usepackage{times,subcaption,pdfpages}
\usepackage{listings}
\usepackage{float}
\usepackage{array} 
\linespread{1}
%\setlength{\parskip}{0.5em}
\usepackage{booktabs}
\usepackage[style=nature,backend=biber]{biblatex}
\usepackage{sectsty}
\usepackage{siunitx}
\usepackage{tabularx}
\usepackage{enumitem}
\usepackage{multirow}
\usepackage{svg}
\chapterfont{\centering}
\sectionfont{\centering}

\begin{document}


\begin{titlepage}
   \begin{center}
       \large
       \vspace*{1cm}

       \textbf{ME5627 Measurements in Thermofluids }

       \vspace{0.5cm}
       
       \textbf{Laboratory Record}
       
       \vspace{6cm}

       \textbf{Submitted By\\~\\}
       {Rakesh S S 132414011 \\Vimal S V 132404001}


       \vspace{4cm}
     
       {\includegraphics[width=4.5cm]{logos/iitpkd_fulllogo_color.pdf} \\}
       
       \vspace{2cm}   
       
       Department Mechanical Engineering\\
       Indian Institute of Technology Palakkad

       %\today           
   \end{center}
\end{titlepage}
\newpage

\chapter*{\Large EXPERIMENT 1}
\setcounter{chapter}{1}
\section*{\normalsize PART A: FAMILIARISATION WITH PC SCOPE AND BREAD BOARD THROUGH
RC CIRCUIT}
\textbf{Aim} – To gain familiarity with Bread Board, oscilloscope and oscilloscope data logger through RC circuit\\
\\
\textbf{Apparatus} – Bread board, wires, Oscilloscope, Resistor, capacitor\\
\\
\textbf{Details of Bread Board:}

\begin{figure}[H]
    \centering
    \begin{subfigure}[b]{0.57\textwidth}
        \centering
        \includegraphics[width=\linewidth]{logos/Breadboard.png}
        \caption{Schematic of the Bread Board}
        \label{fig:Breadboard}
    \end{subfigure}
    \\
    \begin{subfigure}[b]{0.43\textwidth}
        \centering
        \includegraphics[width=\textwidth]{logos/Breadboard_connections.png}
        \caption{Equipotential lines}
        \label{fig:Breadboard_connections}
    \end{subfigure}
    \caption{Breadboard familiarisation}
    \label{fig:Breadboard familiarisation}
\end{figure}

\begin{itemize}
\item Hold the breadboard such that the horizontal side is longer than the vertical side.
\item There are four independent portions (A, B, C and D) created by three notches as in Fig.~\ref{fig:Breadboard}. Here, independence means portions A, B, C and D are not connected with each other
\end{itemize}

\textbf{\textit{Portion A and D}}
\begin{itemize}
\item Each portion of A and D is further subdivided into two regions as A1 and A2 shown in Fig.~\ref{fig:Breadboard familiarisation}.
\item Each horizontal line (shown through arrows in the Fig.~\ref{fig:Breadboard}) of A1, A2, D1 and D2 represent a single potential. Usually, as a general practice, these points are considered as ground for a given built circuit.
\end{itemize}

\textbf{\textit{Portion B and C}}
\begin{itemize}
\item In portion B and C, each vertical line (shown through arrows in Fig.~\ref{fig:Breadboard}) of B and C represents a single potential.
\item The above explanation can be better understood through Fig.~\ref{fig:Breadboard_connections} showing same potential lines.
\end{itemize}

\textbf{\textit{Details of oscilloscope probe}}\\

Fig.~\ref{fig:Oscilloscope_probe} shows a typical oscilloscope probe.

\begin{itemize}
\item The Oscilloscope probe is a passive connector which is used to connect oscilloscope to the electrical circuit. It consists of three parts Retractable hook tip (or crocodile clip), a Crocodile clip and BNC connector.
\item BNC connector: It is used to connect oscilloscope to the probe.
\item Retractable Hook tip: It is used to connect the node of the electrical circuit to the oscilloscope where there is a need to measure the signal.
\item Crocodile clip: It is connected to the ground of the electrical circuit in the bread board.
\item Make sure the RED slider on the probe should be on the X1 position only. X1 means amplification is unity.
\end{itemize}


\begin{figure}[H]
	\centering
	\includegraphics[width=0.6\linewidth]{logos/Oscilloscope_probe.png}
	\caption{PC scope probe}
	\label{fig:Oscilloscope_probe}
\end{figure}

\begin{figure}[H]
	\centering
	\includegraphics[width=0.6\linewidth]{logos/Oscilloscope.png}
	\caption{Four channel Oscilloscope with digital storage}
	\label{fig:Oscilloscope}
\end{figure}


Fig.~\ref{fig:Oscilloscope} shows a four channel Oscilloscope with digital data storage system. There are four
channels which displays the input/output signal. Input and output probes may be connected to any of these
four channels to visualize and analyze the waveforms of the signal. There are scaling, vertical and
horizontal controls to adjust the waveform in order to have better visualization. Peak to peak voltage and
time scale can be conveniently read from the monitor using the adjustment knob after pressing the cursor
button.

\begin{figure}[H]
	\centering
	\includegraphics[width=0.7\linewidth]{logos/Function_Generator.png}
	\caption{Function Generator}
	\label{fig:Function_Generator}
\end{figure}

In order to obtain a desired input signal (sinusoidal, square, ramp, pulse or any arbitrary waveform) a function generator can be used (Fig.~\ref{fig:Function_Generator}). The desired amplitude and frequency of the waveform can be obtained using the side controls near the monitor. The output probes may be connected to any of the two channels. One of the crocodile clip of the output probe needs to be grounded and the other end can be given as input signal to generate the signal in the circuit. Also, in order to check whether the function generator is working, the probe may be connected to the Oscilloscope.\\
\\

\textbf{Calculation of Resistance:}
\begin{itemize}
\item Turn the resistor so that the gold or silver stripe is at the right end of the resistor.
\item Look at the color of the first two stripes on the left end. These correspond to the first two digits
of the resistor value. Use the table given below to determine the first two digits.
\item Look at the third stripe from the left. This corresponds to a multiplication value. Find the value
using the table below.
\item Multiply the two digit number from step two by the number obtained from step three. This is the
value of the resistor in ohms. The fourth stripe indicates the accuracy of the resistor.
\end{itemize}

\begin{figure}[H]
	\centering
	\includegraphics[width=0.7\linewidth]{logos/Resistance_colorcode.png}
	\caption{Resistance color code chart}
	\label{fig:Resistance_colorcode}
\end{figure}

\textbf{Sample:}


You are given a resistor whose stripes are colored from left to right as brown, black, orange, gold. Find
the resistance value.
\begin{itemize}
\item The gold stripe is on the right so go to Step Two.
\item The first stripe is brown which has a value of 1. The second stripe is black which has a value of 0.
Therefore the first two digits of the resistance value are 10.
\item The third stripe is orange which means x 1,000.
\item The value of the resistance is found as 10 x 1000 = 10,000 ohms i.e., 10~k$\Omega$ 
\item The gold stripe means the actual value of the resistor may vary by $5\%$ meaning the actual value
will be somewhere between 9,500 ohms and 10,500 ohms. (Since $5\%$ of 10,000 = 0.05 x 10,000 = 500)
\end{itemize}

\textbf{Calculation of Capacitance:}

\textbf{Sample Calculation:}
\begin{itemize}
\item Step 1: In the Fig.~\ref{fig:Ceramic capacitor}, the first two digits from the left indicates the first two digits of the
capacitor value. Here, in this case the first two digits are 1 and 0. Therefore, the first two digits of
the capacitor value is 10.
\item Step 2: The third digit is 4 which means that four zeroes would be followed by 1 (10,000 pF).
\item The value of the capacitor is found out to be the product of the number obtained in step 1 and step
2. That is, 10 $\times$ 10000 = 10,0000 pF i.e 0.1 $\mu$F.
\end{itemize}

\begin{figure}[H]
	\centering
	\includegraphics[width=0.8\linewidth]{logos/Ceramic_capacitor.PNG}
	\caption{Ceramic capacitor}
	\label{fig:Ceramic capacitor}
\end{figure}

\textbf{Circuit Diagram:}\\
\par Figures.~\ref{fig:RC circuit} and~\ref{fig:PC scope} show the RC circuit and PC scope
\begin{figure}[H]
\centering
\begin{minipage}{.5\textwidth}
  \centering
  \includegraphics[width=6cm]{logos/RC_circuit.png}
  \caption{Typical RC circuit.}
  \label{fig:RC circuit}
\end{minipage}%
\begin{minipage}{.5\textwidth}
  \centering
  \includegraphics[width=5cm]{logos/PC_scope.png}
  \caption{PC Scope.}
  \label{fig:PC scope}
\end{minipage}
\end{figure}


\textbf{Experimental Procedure:}
\begin{itemize}
\item Build an RC circuit as shown in the Fig.~\ref{fig:RC circuit} on the bread board.
\item \textit{\textbf{Circuit input to the oscilloscope}}\\
First PC scope probe is taken. BNC connector of this oscilloscope is connected to the input of the oscilloscope shown in Fig.~\ref{fig:PC scope}. Connect the retractable hook tip of the oscilloscope probe (inputprobe) to the open end of the resistor. The crocodile clip of this input PC scope probe is connected to the ground of the bread board.
\item \textit{\textbf{Circuit Output to the oscilloscope}}\\
Second oscilloscope probe is taken. BNC connector of this oscilloscope is connected to the output of the scope shown in Fig.~\ref{fig:PC scope}. Connect the retractable hook tip of the scope probe (output probe) to common junction of resistor and capacitor. The crocodile clip of this input oscilloscope probe is connected to the ground of the bread board.
\item \textit{\textbf{Desired Signal Input to the oscilloscope}}\\
Third oscilloscope probe is taken. BNC connector of this oscilloscope is connected to the signal generator of the oscilloscope shown in Fig.~\ref{fig:PC scope}. Connect the retractable hook tip of the oscilloscope probe (input probe) to the open end of the resistor. The crocodile clip of this input oscilloscope probe is connected to the ground of the bread board.
\end{itemize}

Following flow chart shows the flow of input and ouput signals.

\begin{figure}[H]
	\centering
	\includegraphics[width=0.8\linewidth]{logos/Flow_chart_IO.PNG}
	\caption{Flowchart indicating the flow of input and output signals}
	\label{fig:Flow chart IO}
\end{figure}

\begin{itemize}
\item There is switch on the probe. Make sure that switch is towards ‘X1’ side so that the values are not amplified by the probe.
\item Give sinusoidal signal of very low amplitude (2 V) as input and observe the output.
\item Increase the amplitude of the input signal and observe the output, repeat this step three times.
\item Measure the peak voltage of both input and output signal using meter A and B.
\item Try giving some other type of signal at some other frequency and observe/measure the output.
\end{itemize}

\textbf{Conclusions/Discussions on the results:}
\newpage
\section*{\centering EXPERIMENT 1}
\section*{\normalsize PART B: BASIC WORKING OF INSTRUMENTATION OP AMP (INA 129)}
\textbf{Aim} – To study the basic working of Instrumentation op amp (INA129)\\
\\
\textbf{Apparatus} – INA 129, Bread board, wires, dc power supply, Oscilloscope\\
\begin{figure}[H]
	\centering
	\includegraphics[width=0.8\linewidth]{logos/OP-Amp.PNG}
	\caption{Schematic of the instrumentation operational amplifier with pins (INA 129)}
	\label{fig:OP-Amp}
\end{figure}

\textbf{Experimental Procedure}

\begin{itemize}
\item Make the circuit as shown in the Fig. 1.8 on the bread board.
\item Connect the Retractable hook tip of the input probe and signal generator probe of the oscilloscope at pin 3 and the crocodile clip of both the probe (input and signal generator) at pin 2 of INA 129.
\item Connect the Retractable hook tip of the output probe at pin 6 and the crocodile clip of the output probe at pin 5 of INA 129. Make a virtual ground in the breadboard and connect pin 5 to this virtual ground.
\item Select the value of RG based on desired gain. The relation between $R_G$ and gain/Amplification (G) is given in Table.~\ref{tab:Rc and gain}
\item Connect the resistance RG across pin 1 and 8 of INA 129.
\item D.C power supply is set to 12V whose positive end is connected to pin 7 while its negative end is connected to pin 4.
\item Give sinusoidal signal of 200mV amplitude and the frequency of 300 Hz in the signal generator and observe the output at the interface of oscilloscope in the monitor. Ensure that pin 5 is grounded.
\item Change the value of $R_G$, repeat the above procedure and observe the output.
\item Note down the output in the observation Table.
\end{itemize}

\begin{table}[h!]
\centering
\begin{tabular}{|c|c|c|c|c|c|}
\hline
\multicolumn{6}{|c|}{\textbf{Observation Table:}} \\ \hline
\textbf{Sr.No} & \textbf{$R_G$} & \textbf{Selected Input} & \textbf{Measured Output} & \textbf{Measured Gain (G)} & \textbf{Calculated Gain} \\ \hline
1 & 49.4K & 200mV & 408mV & 2.04 & 2 \\ \hline
2 & 12.35 & 200mV & 1.06V & 5.3 & 5 \\ \hline
3 & 1.008K & 200mV & 10.2V & 51 & 50 \\ \hline
\end{tabular}
\caption{Relation between $R_G$ and Desired gain}
\label{tab:Rc and gain}
\end{table}

\begin{equation*}
    G = 1 + \frac{\SI{49.4}{\kilo\ohm}}{R_G}
\end{equation*}

\begin{table}[ht]
\centering
\begin{tabular}{|c|c|c|}
\hline
\textbf{Desired Gain (V/V)} & \textbf{$R_G$ (\si{\ohm})} & \textbf{Nearest 1\% $R_G$ (\si{\ohm})} \\ \hline
1     & NC     & NC \\ \hline
2     & 49.4K  & 49.9K \\ \hline
5     & 12.35K & 12.4K \\ \hline
10    & 5489   & 5.49K \\ \hline
20    & 2600   & 2.61K \\ \hline
50    & 1008   & 1K    \\ \hline
100   & 499    & 499   \\ \hline
200   & 248    & 249   \\ \hline
500   & 99     & 100   \\ \hline
1000  & 49.5   & 49.9  \\ \hline
2000  & 24.7   & 24.9  \\ \hline
5000  & 9.88   & 9.76  \\ \hline
10000 & 4.94   & 4.87  \\ \hline
\end{tabular}
\caption{Desired Gain and Corresponding $R_G$ Values}
\label{table:rg_values}
\end{table}

\begin{figure}[H]
	\centering
	\includegraphics[width=0.5\linewidth]{logos/OP-Amp_gain_circuit.PNG}
	%\caption{Flowchart indicating the flow of input and output signals}
	\label{fig:op-amp_gain}
\end{figure}

\textbf{Conclusions/Discussion on the Results:}


\newpage

\chapter*{\Large EXPERIMENT 2}
\setcounter{chapter}{2}
\setcounter{table}{0}
\setcounter{figure}{0}
\section*{\normalsize DYNAMIC RESPONSE OF A FIRST ORDER SYSTEM USING R-C CIRCUIT}
\textbf{Aim}

\begin{itemize}
\item To study and understand the behavior of a first order system simulated by an RC circuit using an oscilloscope-cum-signal generator with sinusoidal and ramp inputs. The output amplitude and the phase difference are studied with various frequencies of the input signal.
\item Find the response characteristic of RC circuit for a generated square wave using Fourier series as input compare with the theoretical response
\end{itemize} 
\textbf{Apparatus} – Resistor, capacitor, breadboard, oscilloscope\\
\\
\textbf{Procedure:}
\begin{itemize}
\item Construct the circuit as shown in Fig.~\ref{fig:RC_circuit_2} with given electronic components.
\begin{figure}[H]
	\centering
	\includegraphics[width=0.3\linewidth]{logos/RC_circuit_2.PNG}
	\caption{R-C Circuit}
	\label{fig:RC_circuit_2}
\end{figure}
\item Chose appropriate resistance and capacitance ($R = 1 k\si{\ohm}$ and $C = 0.1 \mu F$). Calculate the cut off frequency $f = \frac{1}{2\pi RC}$
\item Connect the oscilloscope-cum-signal generator output probe to supply the input as shown. Connect “Channel A” probes of the oscilloscope-cum-signal generator to the input probe and “Channel B” across the capacitor to extract the output.
\item The input to the circuit will be sinusoidal and square signals of constant amplitude (~ 5 volts peak to peak) and variable frequency in order to obtain the frequency response characteristics.
\item Apply a sinusoidal wave as the input to the RC circuit. Study the output across the capacitor on the screen and make quantitative observations (see Table.~\ref{tab:sine_input}) regarding the amplitude and phase of input and output waveform.
\item Using the oscilloscope function in the PC software, Export the input and output signals to a “.csv” file for further calculation. Measure the amplitude and phase difference between input and output in MS Excel.
\item Take three different readings above, three readings below and one at the cut off frequency.
\item Calculate the static gain in each case, plot the frequency response of the given RC circuit for the static gain, and phase difference on the graph sheets.
\item Repeat the experiment for square wave as the input to the RC circuit. However, select input frequencies such that they are lower than the cut off frequency.

\textbf{Observation Table:}

\begin{table}[H]
\centering
\caption{Sinusoidal input}
\begin{tabularx}{\textwidth}{|p{0.5cm}|X|X|X|X|X|X|X|X|}
\hline
\textbf{Sr. No.} & \textbf{Frequency (Hz)} & \textbf{Input peak to peak (Volts)} & \textbf{Output peak to peak (Volts)} & \textbf{Measured gain voltage/input} $\left( \frac{q_o}{K_q} \right)$ & \textbf{Static Output/Input gain} & \textbf{Calculated static gain} & \textbf{Measured Phase difference $\phi$} & \textbf{Calculated phase difference} \\
\hline
1 &  &  &  &  &  &  &  &  \\
\hline
2 &  &  &  &  &  &  &  &  \\
\hline
3 &  &  &  &  &  &  &  &  \\
\hline
4 &  &  &  &  &  &  &  &  \\
\hline
5 &  &  &  &  &  &  &  &  \\
\hline
6 &  &  &  &  &  &  &  &  \\
\hline
7 &  &  &  &  &  &  &  &  \\
\hline
\end{tabularx}
\label{tab:sine_input}
\end{table}

\item Plot decibel value variation with frequency (Bode Plot) and measure the slope
\begin{equation*}
Decibel~Value = dB = 20log\bigg\lvert\frac{q_0(\omega)}{q_i(\omega)}\bigg\rvert
\end{equation*}
\item Take the input signal and output signal (generated from the circuit) from the generated .csv file. Calculate the output signal using the frequency response characteristics of first order system.
\end{itemize}

\begin{table}[H]
\centering
\caption{Square wave input using PC scope}
\begin{tabularx}{\textwidth}{|c|c|X|X|X|}
\hline
\textbf{Sr. No.} & \textbf{Frequency (Hz)} & \textbf{Input peak to peak (Volts)} & \textbf{Measured Output Peak to peak (Volts)} & \textbf{Measured Output Peak to peak / Measured input peak to peak (Volts)} \\ \hline
1 &  &  &  &  \\ \hline
2 &  &  &  &  \\ \hline
3 &  &  &  &  \\ \hline
4 &  &  &  &  \\ \hline
5 &  &  &  &  \\ \hline
6 &  &  &  &  \\ \hline
\end{tabularx}
\end{table}

\noindent Comment on the comparison of measured response characteristic of RC circuit for a given of square wave input in terms of Fourier series with the calculated response.

\noindent Comment on RC circuit when used as High pass and Low pass filter.

\noindent \textbf{Conclusions/Discussion on the Results:}

\chapter*{\Large EXPERIMENT 3}
\setcounter{chapter}{3}
\setcounter{table}{0}
\setcounter{figure}{0}
\section*{\normalsize FREQUENCY RESPONSE OF A SECOND ORDER SYSTEM USING RLC CIRCUIT}
\textbf{Aim}

\begin{itemize}
\item To study the frequency response of a series RLC circuit for different damping ratios.
\item To determine the resonance frequency in a series RLC circuit and compare its value with the theoretical value
\end{itemize} 
\textbf{Apparatus} – Bread-board, variable resistor (potentiometer), capacitor, inductor, multi-meter, Oscilloscope and connecting wires.\\
\\
\textbf{Procedure:}
\begin{enumerate}
\item Determination of resonance frequency
\begin{enumerate}[label=\roman*]
\item The function generator frequency is set to 50 Hz before making the connection. The signal generator output is set to 3V (peak to peak).
\item Using the breadboard and wire leads connect the resistor, capacitor and inductor along with the output of the function generator to construct the circuit shown in Fig.~\ref{fig:LCR_circuit}. Note that the peak to peak voltage is measured across the resistor using the oscilloscope.
\item Make sure that the resistance is less than 447 $\si{\ohm}$ i.e. the system is underdamped , for Part 1 of the experiment. All the components are connected in series with the supply given by the function generator. All the ground connections are made at one end of the resistor. \label{step4}

\begin{figure}[H]
	\centering
	\includegraphics[width=0.4\linewidth]{logos/LCR_circuit_series.PNG}
	\caption{Series RLC circuit for resonance frequency}
	\label{fig:LCR_circuit}
\end{figure}

\item Use following relation to compute the expected resonance frequency and record your results in Table~\ref{tab:O/p volt vs freq}\label{step4}
\begin{equation*}
\omega_n = \frac{1}{\sqrt{LC}}
\end{equation*}

\item Set the function generator frequency to 100~Hz and record the peak to peak voltage from the oscilloscope. Then, adjust the excitation frequency to 200~Hz and record the voltage. Adjust the frequency to 300~Hz and record the voltage. Continue varying the excitation frequency to discrete values below the expected resonance frequency computed in step (\ref{step4}). Record the voltage for each of these values.
\item Determine the experimental value for resonance frequency by finding the frequency that produces the largest voltage on the oscilloscope. Record this frequency and voltage.
\item Record the voltage for frequency values that are above the resonance frequency determined in step (\ref{step4}).
\item Plot a graph between voltage and frequency ($f$) and determine the resonance frequency of the circuit.
\end{enumerate}
\item Frequency response of RLC circuit for different damping ratios
\begin{figure}[H]
	\centering
	\includegraphics[width=0.4\linewidth]{logos/LCR_Damping.PNG}
	\caption{Series RLC circuit for different damping ratios}
	\label{fig:LCR_circuit_damping}
\end{figure}
\begin{enumerate}[label=\roman*]
\item Using the breadboard and wire leads connect the resistor, capacitor and inductor along with the output of the function generator to construct the circuit shown in Figure 3.2. Note that the peak to peak voltage is measured across the capacitor using the
oscilloscope.
\item The function generator frequency is set to 100 Hz before making the connection. The signal generator output is set to 3V (peak to peak).
\item Use the same circuit as used in Part (1) of this experiment. In this case, the output voltage is measured across the capacitor, so connect the terminals of the oscilloscope across the capacitor.
\item The damping ratio $\zeta$ is given by
\begin{equation*}
\zeta = \frac{R}{2}\sqrt{\frac{C}{L}}
\end{equation*}
Vary the value of the resistance such that $L$ takes different values, so that overdamping, underdamping and critical damping in the circuit is observed
\item Adjust the resistance such that the system corresponds to an underdamped system. Apply a frequency of 100 Hz and record the input and output voltage. Change the excitation frequency as integer multiples of resonance frequency ($\omega = \omega_n, \omega_{2n}, \omega_{3n}, \omega_{4n}$) and obtain the input and output voltages.\label{ste4}
\item Repeat step \ref{ste4} for different damping ratios. Take 3 sets of readings. One reading for under-damped and over-damped system, and one reading for critically damped case. (You can change the damping ratio by changing the resistance value of the variable resistance).
\item Plot a graph between $Vo/Vi$ and $\omega/ \omega_n$ for different values of $L$.
\item Compare the experimental and theoretical behavior of the system. Theoretically, the ratio $Vo/Vi$ of a second order system is given as
\begin{equation*}
\bigg\lvert{\frac{V_0}{V_i}}\bigg\rvert = \frac{1}{\sqrt{(1-\frac{\omega^2}{\omega_n^2}) + (\frac{2\zeta\omega}{\omega_n})^2}}
\end{equation*}
\end{enumerate}
\end{enumerate}

\textbf{Observations :}

Inductance = 50$mH$

Capacitance = 0.1$\mu F$

Resitance = 1-20k$\si{\ohm}$\\
\\

\textbf{Calculation of Resonance frequency}

Theoretical resonance frequency = .....Hz

Expeimental resonance frequency = .....Hz

Percentage difference = .....%

\begin{table}[h!]
\centering
\caption{Variation of Output Voltage with Frequency}
\begin{tabular}{|c|c|c|c|}
\hline
\textbf{S/ No.} & \textbf{Frequency $f$ (Hz)} & \textbf{Input Voltage $V_i$ (V)} & \textbf{Output Voltage $V_o$ (V)} \\ \hline
1 &  &  &  \\ \hline
2 &  &  &  \\ \hline
3 &  &  &  \\ \hline
4 &  &  &  \\ \hline
5 &  &  &  \\ \hline
6 &  &  &  \\ \hline
7 &  &  &  \\ \hline
8 &  &  &  \\ \hline
9 &  &  &  \\ \hline
10 &  &  &  \\ \hline
11 &  &  &  \\ \hline
12 &  &  &  \\ \hline
\end{tabular}
\label{tab:OP_vs_freq}
\end{table}

\begin{table}[H]
\centering
\begin{tabular}{|c|c|c|c|c|c|c|}
\hline
\textbf{Sl No.} & \textbf{$\xi$} & \textbf{Type of the system} & \textbf{f} & \textbf{$V_o$ (mV)} & \textbf{Theoretically calculated $V_o$ (mV)} & \textbf{Deviation (\%)} \\ \hline
1  &   &   &   &   &   &   \\ \hline
2  &   &   &   &   &   &   \\ \hline
3  &   &   &   &   &   &   \\ \hline
4  &   &   &   &   &   &   \\ \hline
5  &   &   &   &   &   &   \\ \hline
6  &   &   &   &   &   &   \\ \hline
7  &   &   &   &   &   &   \\ \hline
8  &   &   &   &   &   &   \\ \hline
9  &   &   &   &   &   &   \\ \hline
10 &   &   &   &   &   &   \\ \hline
11 &   &   &   &   &   &   \\ \hline
12 &   &   &   &   &   &   \\ \hline
\end{tabular}
\caption{Variation of output voltage with frequency for different $\xi$}
\end{table}

\textbf{Conclusions/Discussions on the results:}
\newpage

\chapter*{\Large EXPERIMENT 4}
\setcounter{chapter}{4}
\setcounter{table}{0}
\setcounter{figure}{0}
\section*{\normalsize PART A: APPLICATION OF OP-AMP AS NON-INVERTING AMPLIFIER}
\textbf{Aim} - To set-up the circuit of op-amp as non-inverting amplifier and measure its gain.\\
\\
\textbf{Apparatus} – IC 741, Resistors – 1 k$\si{\ohm}$, 100 $\si{\ohm}$, 10 k$\si{\ohm}$, Regulated DC power supply, Oscilloscope, Multimeter.\\
\\
\textbf{Circuit Diagram}
\begin{figure}[H]
	\centering
	\includegraphics[width=0.5\linewidth]{logos/Non_inv_amp.png}
	\caption{Circuit of Non-inverting amplifier}
	\label{fig:Non_inv_amp}
\end{figure}

\begin{figure}[H]
	\centering
	\includegraphics[width=0.4\linewidth]{logos/IC741.png}
	\caption{Schematic of IC741 (Op-AMP)}
	\label{fig:IC741_op-amp}
\end{figure}

\textbf{Design Procedure:}
\begin{itemize}
\item Select the desired gain of the amplifier.
\item The ideal gain of non-inverting amplifier is given by
\begin{equation*}
A_F = 1+\frac{R_f}{R_i}
\end{equation*}
\item For Ex: Gain = 11; R1 = 1 k$\si{\ohm}$ RF = 10 k$\si{\ohm}$ OR R1 = 100 $\si{\ohm}$ RF = 1 k$\si{\ohm}$ (Gain is dependent on the ratio of resistors, not the absolute values)
\item The exact gain of non-inverting amplifier is given by
\begin{equation*}
A_F = \frac{V_o}{V_i} = \frac{A(R_1 + R_f)}{R_1 + R_f + AR_1}
\end{equation*}
\end{itemize}

\textbf{Experimental Procedure:}
\begin{itemize}
\item Connect the circuit as shown in Fig.~\ref{fig:Non_inv_amp}.
\item Apply +12~V to pin 7 and –12~V to pin 4. Connect common terminal of power supply to ground on the breadboard.
\item Apply an A.C voltage of 0.2~V using Oscilloscope to the pin-3 of IC. Measure the output using oscilloscope.
\item Increase voltage in steps of 0.2~V up to 1.0~V and measure the output. Note down the output in observation table.
\item Change the value of all resistance such that the signal after amplification does not exceed 10~V.
\item Apply a D.C voltage of 1~V using Oscilloscope to the pin-3 of IC. Measure the output.
\item Increase voltage in steps of 0.5~V up to 2.0~V and measure the output. Note down the output in observation table.
\end{itemize}

\begin{table}[h!]
\centering
\caption{Non-inverting Amplifier gain}
\begin{tabularx}{\textwidth}{|p{0.5cm}|p{1cm}|X|X|X|X|X|X|X|}
\hline
\textbf{S. No.} & \textbf{Type of power supply to input} & \textbf{Non-Inverting input voltage (V)} & \textbf{Output voltage (V)} & \textbf{Measured Gain} & \textbf{Ideal Gain} & \textbf{Deviation} & \textbf{Exact Gain} & \textbf{Deviation} \\ \hline
1  & \centering A.C &   &   &   &   &   &   &   \\ \hline
2  & \centering A.C &   &   &   &   &   &   &   \\ \hline
3  & \centering A.C &   &   &   &   &   &   &   \\ \hline
4  & \centering A.C &   &   &   &   &   &   &   \\ \hline
5  & \centering A.C &   &   &   &   &   &   &   \\ \hline
6  & \centering D.C &   &   &   &   &   &   &   \\ \hline
7  & \centering D.C &   &   &   &   &   &   &   \\ \hline
8  & \centering D.C &   &   &   &   &   &   &   \\ \hline
9  & \centering D.C &   &   &   &   &   &   &   \\ \hline
10 & \centering D.C &   &   &   &   &   &   &   \\ \hline
\end{tabularx}
\label{tab:Non-inverting Amplifier gain}
\end{table}

\textbf{Conclusions/Discussion On The Results:}

\newpage
\section*{\centering EXPERIMENT 4}
\section*{\normalsize PART B: APPLICATION OF OP-AMP AS INVERTING AMPLIFIER}
\textbf{Aim} – To set-up the circuit of op-amp as inverting amplifier and measure its gain\\
\\
\textbf{Apparatus} – IC 741, Resistors – 1 k$\si{\ohm}$, 100 $\si{\ohm}$, 10 k$\si{\ohm}$, Regulated DC powersupply, Oscilloscope, Multimeter.\\
\\
\textbf{Circuit Diagram}
\begin{figure}[H]
	\centering
	\includegraphics[width=0.45\linewidth]{logos/Inv_amp.png}
	\caption{Circuit of inverting amplifier}
	\label{fig:Inv_amp}
\end{figure}

\textbf{Design Procedure}
\begin{itemize}
\item Select the desired gain of the amplifier.
\item The ideal gain of non-inverting amplifier is given by
\begin{equation*}
A_F = \frac{V_o}{V_i} = \frac{A R_F}{A R_i} = \frac{R_F}{R_i}
\end{equation*}

\item For Ex: Gain = -10; R1 = 1 k$\si{\ohm}$ RF = 10 k$\si{\ohm}$ OR R1 = 100 $\si{\ohm}$ RF = 1 k$\si{\ohm}$ (Gain is dependent on the ratio of resistors, not the absolute values)
\item The exact gain of non-inverting amplifier is given by
\begin{equation*}
\frac{V_o}{V_i} = - \frac{A R_F}{R_1 + R_F + A R_i}
\end{equation*}
\end{itemize}

\textbf{Experimental Procedure}
\begin{itemize}
\item Connect the circuit as shown in Fig.~\ref{fig:Inv_amp}
\item Apply +12~V to pin 7 and –12~V to pin 4. Connect common terminal of power supply to ground on the breadboard.
\item Apply an A.C voltage of 0.2~V using PC Oscilloscope to the pin-2 of IC. Measure the output.
\item Increase voltage in steps of 0.2~V up to 1.0~V and measure the output. Note down the output in observation table.
\item Change the value of all resistance such that the signal after amplification does not exceed 10~V.
\item Apply a D.C voltage of 1~V using Oscilloscope to the pin-2 of IC. Measure the output.
\item Increase voltage in steps of 0.5~V up to 2.0~V and measure the output. Note down the output in observation table.
\end{itemize}

\begin{table}[H]
\centering
\caption{Inverting Amplifier Gain}
\begin{tabularx}{\textwidth}{|p{0.5cm}|X|X|X|X|X|X|X|X|X|}
\hline
\textbf{Sr. No.} & \textbf{Type of Input} & \textbf{Inverting input voltage (V)} & \textbf{Output voltage (V)} & \textbf{Measured Gain} & \textbf{Ideal Gain} & \textbf{Deviation} & \textbf{Exact Gain} & \textbf{Deviation} \\ \hline
1  & \centering A.C 		  &   &   &   &   &   &   &   \\ \hline
2  & \centering A.C          &   &   &   &   &   &   &   \\ \hline
3  & \centering A.C          &   &   &   &   &   &   &   \\ \hline
4  & \centering A.C          &   &   &   &   &   &   &   \\ \hline
5  & \centering A.C		  &   &   &   &   &   &   &   \\ \hline
6  & \centering D.C          &   &   &   &   &   &   &   \\ \hline
7  & \centering D.C          &   &   &   &   &   &   &   \\ \hline
8  & \centering D.C          &   &   &   &   &   &   &   \\ \hline
9  & \centering D.C          &   &   &   &   &   &   &   \\ \hline
10 & \centering D.C          &   &   &   &   &   &   &   \\ \hline
\end{tabularx}
\label{tab:Inverting amplifier gain}
\end{table}

\textbf{Conclusions/Discussion On The Results:}

\newpage

\section*{\centering EXPERIMENT 4}
\section*{\normalsize PART C: APPLICATION OF OP-AMP AS DIFFERNTIAL AMPLIFIER}
\textbf{Aim} – To set-up the circuit of op-amp as differential amplifier and measure its gain\\
\\
\textbf{Apparatus} – IC 741, Resistors – 1 k$\si{\ohm}$, 100 $\si{\ohm}$, 10 k$\si{\ohm}$, Regulated DC power supply, Multimeter.\\
\\
\textbf{Circuit Diagram}

\begin{figure}[H]
	\centering
	\includegraphics[width=0.5\linewidth]{logos/Differential_amp.png}
	\caption{Circuit of differential amplifier}
	\label{fig:Diff_amp}
\end{figure}

\textbf{Design Procedure:}
\begin{itemize}
\item Select the desired gain of the amplifier.
\item The ideal gain of the differential amplifier is given by
\begin{equation*}
A_D = - \frac{V_o}{V_{xy}} = - \frac{R_F}{R_1}
\end{equation*}
\item Use $R_2 = R_1$ and $R_F = R_3$
\end{itemize}

\textbf{Experimental Procedure:}
\begin{itemize}
\item Connect the circuit as shown in Fig.~\ref{fig:Diff_amp}.
\item Apply +12~V to pin 7 and –12~V to pin 4. Connect common terminal of power supply to ground on the breadboard.
\item Apply two different voltages $V_x$ and $V_y$ and measure the output. Make sure that the difference in the voltage should not exceed 10~V.
\item Repeat the above step for different $V_x$ and $V_y$
\end{itemize}

\textbf{Observations:}

\begin{table}[H]
\centering
\caption{Differential Amplifier Gain}
\begin{tabularx}{\textwidth}{|p{0.5cm}|X|X|X|X|X|}
\hline
\textbf{Sr No.} & \textbf{Input voltage ($V_x$)} & \textbf{Input voltage ($V_y$)} & \textbf{Measured Output voltage (V)} & \textbf{Calculated output voltage} & \textbf{Deviation} \\ \hline
1 &   &   &   &   &   \\ \hline
2 &   &   &   &   &   \\ \hline
3 &   &   &   &   &   \\ \hline
4 &   &   &   &   &   \\ \hline
5 &   &   &   &   &   \\ \hline
\end{tabularx}
\label{tab:Differential Amplifier Gain}
\end{table}

\textbf{Conclusions/Discussion on the results:}

\newpage

\chapter*{\Large EXPERIMENT 5}
\setcounter{chapter}{5}
\setcounter{table}{0}
\setcounter{figure}{0}
\section*{\normalsize PART A: OPERATIONAL AMPLIFIER AS A DIFFERENTIATOR}
\textbf{Aim} - To study the application of an op-amp as differentiator..\\
\\
\textbf{Apparatus} – IC 741, Resistors - 150~$\si{\ohm}$, 1.5~k$\si{\ohm}$, Capacitors – 0.1~$\mu$F, 0.01~$\mu$F, Oscilloscope, Signal Generator, Regulated DC power supply.\\
\\
\textbf{Circuit Diagram}

\begin{figure}[H]
	\centering
	\includegraphics[width=0.5\linewidth]{logos/Differentor_circuit.png}
	\caption{Differentiator Circuit}
	\label{fig:Differentor_circuit}
\end{figure}

\begin{figure}[H]
	\centering
	\includegraphics[width=0.4\linewidth]{logos/IC741.png}
	\caption{Schematic of IC 741(OP-Amp)}
	\label{fig:IC 741(OP-Amp)}
\end{figure}

\textbf{Design Procedure:}
\begin{itemize}
\item Select given input frequency $f_a$ such that,
\begin{equation*}
f_a = \frac{1}{2\pi R_1 C_1}
\end{equation*}
\item Assume $C_1$ and find the value of $R_F$ given by the above equation.
\item \textbf{Sentences to be corrected}
\end{itemize}

\textbf{Experimental Procedure:}
\begin{itemize}
\item Connect the circuit as shown in Fig.~\ref{fig:Differentor_circuit}
\item Apply Square/Sine wave input of desired frequency and magnitude (4~V (peak to peak) and 1~kHz).
\item Observe the output voltage at pin 6 of op-amp.
\end{itemize}

\textbf{Observations:}

\par The output waveform from a differentiator is observed by applying square and sine wave signals.

\begin{table}[H]
\centering
\caption{Differentiator Input and Output observations}
\begin{tabularx}{\textwidth}{|X|X|X|X|X|X|X|X|}
\hline
\multicolumn{2}{|c|}{\textbf{Input-Square Wave}} & \multicolumn{2}{c|}{\textbf{Output-Spikes}} & \multicolumn{2}{c|}{\textbf{Input-Sine Wave}} & \multicolumn{2}{c|}{\textbf{Output-Cosine Wave}} \\ \hline
\textbf{Amplitude (V\textsubscript{P-P})} & \textbf{Time period (ms)} & \textbf{Amplitude (V\textsubscript{P-P})} & \textbf{Time period (ms)} & \textbf{Amplitude (V\textsubscript{P-P})} & \textbf{Time period (ms)} & \textbf{Amplitude (V\textsubscript{P-P})} & \textbf{Time period (ms)} \\ \hline
& & & & & & & \\ \hline
& & & & & & & \\ \hline
& & & & & & & \\ \hline
& & & & & & & \\ \hline
\end{tabularx}
\label{tab:Differentiator Input and Output observations}
\end{table}

\textbf{Sample Calculation:}\\
\par
For $T$ = 1~ms, $f_a = \frac{1}{T}$ = 1~kHz\\
\par
$f_a = \frac{1}{2\pi R_F C_1}$ = 1~kHz\\
\par
Select $C_1$ = 0.1$\mu$F, then $R_F$ = 1.5~K$\si{\ohm}$\\
\par
$f_b$ = 10$f_a$ = $\frac{1}{2\pi R_1 C_1}$\\
\par
For $C_1$ = 0.1$\mu$F, $R_1$ = 150~$\si{\ohm}$\\
\par
$R_1 C_1$ = $R_F C_F$\\
\par
$C_F$ = 0.01~$\mu$F
\\
\par
\textbf{Conclusions/Discussion on the results:}
\end{document}